\nonstopmode{}
\documentclass[a4paper]{book}
\usepackage[times,inconsolata,hyper]{Rd}
\usepackage{makeidx}
\makeatletter\@ifl@t@r\fmtversion{2018/04/01}{}{\usepackage[utf8]{inputenc}}\makeatother
% \usepackage{graphicx} % @USE GRAPHICX@
\makeindex{}
\begin{document}
\chapter*{}
\begin{center}
{\textbf{\huge Package `LandS'}}
\par\bigskip{\large \today}
\end{center}
\ifthenelse{\boolean{Rd@use@hyper}}{\hypersetup{pdftitle = {LandS: Biostatistic Tools for DCP}}}{}
\begin{description}
\raggedright{}
\item[Type]\AsIs{Package}
\item[Title]\AsIs{Biostatistic Tools for DCP}
\item[Version]\AsIs{1.7}
\item[Author]\AsIs{Stefano Bergamini and Luca Lalli }
\item[Maintainer]\AsIs{Stefano Bergamini }\email{stefano.bergamini@istitutotumori.mi.it}\AsIs{}
\item[Description]\AsIs{This package provides useful functions in daily clinical pratice for biostatistics}
\item[License]\AsIs{MIT + file LICENSE}
\item[Encoding]\AsIs{UTF-8}
\item[LazyData]\AsIs{true}
\item[RoxygenNote]\AsIs{7.3.2}
\item[URL]\AsIs{}\url{https://stheberga.github.io/LandS/}\AsIs{}
\item[Imports]\AsIs{ggplot2,
dplyr,
ggpubr,
progress,
ggh4x,
grid,
gridExtra,
svMisc,
officer,
rvg,
progress,
PMCMRplus}
\item[Suggests]\AsIs{knitr,
rmarkdown}
\item[VignetteBuilder]\AsIs{knitr}
\end{description}
\Rdcontents{Contents}
\HeaderA{Boxplot\_LB}{This function creates a list of boxplot}{Boxplot.Rul.LB}
%
\begin{Description}
This function creates a list of boxplot
\end{Description}
%
\begin{Usage}
\begin{verbatim}
Boxplot_LB(
  data,
  variables,
  group,
  ID_lines = FALSE,
  Posthoc = FALSE,
  Point = F,
  Median_line = F,
  rm.outliers = F,
  alpha_box = 0.1,
  width_box = 0.2,
  size_median_line = 0.8,
  col_median_line = "red",
  lwd_box = 0.1,
  lwd_ID_line = 0.2,
  alpha_ID_line = 0.3,
  alpha_point = 0.3,
  size_point = 0.3,
  Test_results = NULL,
  threshold_posthoc = 0.1,
  posthoc_test_size = 3.88,
  bracket_shorten = 0,
  bracket.nudge.y = 0,
  Overall = F,
  notch = F,
  notchwidth = 0.5,
  axis_y_title = NULL,
  axis_x_title = NULL,
  size_axis_x = 6,
  size_axis_y = 6,
  ID = "ID",
  legend_cod = NULL,
  breaks_axis_x = levels(data[, group]),
  labels_axis_x = levels(data[, group]),
  grid = TRUE,
  PPTX = FALSE,
  pptx_width = 7.5,
  pptx_height = 5.5,
  extra = F,
  extra_text = NULL,
  palette_boxplot = rep("salmon", nlevels(data[, group])),
  label_legend_title = paste0("Boxplots by ", group, "\n", format(Sys.Date(),
    "%d/%m/%Y")),
  palette_title = "black",
  size_title = 3,
  size_legend_title = 3,
  size_legend_text = 3,
  size_legend_circle = 4,
  target = paste0(path_out, "/Boxplot.pptx"),
  ratio = 1,
  telegram = "none"
)
\end{verbatim}
\end{Usage}
%
\begin{Arguments}
\begin{ldescription}
\item[\code{data}] dataframe

\item[\code{variables}] vector containing all variables of interest

\item[\code{group}] factor variable splitting the data

\item[\code{ID\_lines}] whether to print the lines for paired observations

\item[\code{Posthoc}] whether to display Posthoc tests

\item[\code{Point}] whether to display observation points

\item[\code{Median\_line}] whether to display the line connecting medians

\item[\code{rm.outliers}] whether to remove outliers from display

\item[\code{alpha\_box}] alpha parameter for the boxes

\item[\code{width\_box}] box's width

\item[\code{size\_median\_line}] median linewidth

\item[\code{col\_median\_line}] Colour for the median line. Default = red

\item[\code{lwd\_box}] box linewidth

\item[\code{lwd\_ID\_line}] linewidth for paired observations

\item[\code{alpha\_ID\_line}] alpha for paired observations

\item[\code{alpha\_point}] alpha for points

\item[\code{size\_point}] size for points

\item[\code{Test\_results}] dataframe for global and posthoc tests, see cont\_var\_test\_LB

\item[\code{threshold\_posthoc}] threshold for displaying posthoc tests

\item[\code{bracket\_shorten}] [0,1] width of the bracket

\item[\code{bracket.nudge.y}] Vertical adjustment to nudge brackets by. Useful to move up or move down the bracket. If positive value, brackets will be moved up; if negative value, brackets are moved down.

\item[\code{notch}] add notch to boxplot

\item[\code{notchwidth}] width of the notch

\item[\code{axis\_y\_title}] axis y title

\item[\code{axis\_x\_title}] axis x title

\item[\code{size\_axis\_x}] axis y title dimension

\item[\code{size\_axis\_y}] axis x title dimension

\item[\code{ID}] ID variable
\end{ldescription}
\end{Arguments}
%
\begin{Value}
Una lista di boxplot
\end{Value}
\HeaderA{cont\_var\_test\_LB}{Test for continuous variables splitted by categories}{cont.Rul.var.Rul.test.Rul.LB}
%
\begin{Description}
The most powerful function ever created. You can perform the 4 major tests and the posthoc tests for Friedman and Kruskal-Wallis.
If you are dumb (option dumb = T) you can also perform posthoc tests without correcting for test multiplicity.
Please do not try this at home/work and consider asking a statistician before performing any test.
\end{Description}
%
\begin{Usage}
\begin{verbatim}
cont_var_test_LB(
  data,
  variables,
  paired = FALSE,
  group,
  dumb = FALSE,
  ID = "ID",
  num_dec = 2,
  p.adjust.method = NULL,
  excel = F,
  excel_path = paste0(path_out, "/Results.xlsx"),
  telegram = "none"
)
\end{verbatim}
\end{Usage}
%
\begin{Arguments}
\begin{ldescription}
\item[\code{data}] dataframe

\item[\code{variables}] vector containing all variables of interest

\item[\code{paired}] FALSE/TRUE

\item[\code{group}] factor variable splitting the data

\item[\code{dumb}] FALSE are you dumb? Hope not

\item[\code{ID}] ID variabl (Default = "ID")

\item[\code{num\_dec}] Decimal number for mean and SD (Default = 2)

\item[\code{p.adjust.method}] correction method, a character string. Can be abbreviated.

\item[\code{excel}] export fuction results as multiple Excel sheets

\item[\code{excel\_path}] path where you want your Excel

\item[\code{telegram}] send a telegram message
\end{ldescription}
\end{Arguments}
%
\begin{Value}
Una lista con dataset
\end{Value}
%
\begin{Examples}
\begin{ExampleCode}
cont_var_test_LB(data = iris, variables = c("Sepal.Length", "Sepal.Width"), group = "Species", paired = FALSE)
\end{ExampleCode}
\end{Examples}
\HeaderA{correlazioni\_LB}{This function computes the correlation coefficients and prints the pairs from the heightest coefficient}{correlazioni.Rul.LB}
%
\begin{Description}
This function computes the correlation coefficients and prints the pairs from the heightest coefficient
\end{Description}
%
\begin{Usage}
\begin{verbatim}
correlazioni_LB(
  data,
  variables,
  method = "spearman",
  rho_dec = 3,
  pval_dec = 4,
  excel = FALSE,
  excel_path = paste0(path_output, "/Results.xlsx")
)
\end{verbatim}
\end{Usage}
%
\begin{Arguments}
\begin{ldescription}
\item[\code{data}] dataframe

\item[\code{variables}] vector of numeric variables to be computed the correlation

\item[\code{method}] method to compute the correlation coefficient (Default = "spearman")

\item[\code{rho\_dec}] number of decimal for rho (Default = 3)

\item[\code{pval\_dec}] number of decimal for the pvalue (Default = 4)

\item[\code{excel}] export fuction results as multiple Excel sheets

\item[\code{excel\_path}] path where you want your Excel
\end{ldescription}
\end{Arguments}
%
\begin{Value}
Una lista con dataset
\end{Value}
\HeaderA{Distribution\_LB}{Distribution function}{Distribution.Rul.LB}
%
\begin{Description}
Distribution function
\end{Description}
%
\begin{Usage}
\begin{verbatim}
Distribution_LB(data, var, split = FALSE, split_rule = NULL)
\end{verbatim}
\end{Usage}
%
\begin{Arguments}
\begin{ldescription}
\item[\code{data}] a dataframe

\item[\code{var}] variable

\item[\code{split}] does the variable need to be splitted

\item[\code{split\_rule}] the splitting rule
\end{ldescription}
\end{Arguments}
%
\begin{Value}
a plot with the variable
\end{Value}
%
\begin{Examples}
\begin{ExampleCode}
Distribution_LB(data = mtcars, var = "mpg", split = TRUE, split_rule = 23)
\end{ExampleCode}
\end{Examples}
\HeaderA{filename\_LB}{This function returns the filename to be outputted}{filename.Rul.LB}
%
\begin{Description}
This function returns the filename to be outputted
\end{Description}
%
\begin{Usage}
\begin{verbatim}
filename_LB(
  filename = "Prova",
  extention = ".png",
  output = path_output,
  datetime = F
)
\end{verbatim}
\end{Usage}
%
\begin{Arguments}
\begin{ldescription}
\item[\code{filename}] name of the file

\item[\code{extention}] file extention

\item[\code{output}] the main output path

\item[\code{datetime}] whether to print the datetime in a cute format
\end{ldescription}
\end{Arguments}
\HeaderA{formatz\_p}{Function to get a formatted p-value for a number o a vector of numbers}{formatz.Rul.p}
%
\begin{Description}
Function to get a formatted p-value for a number o a vector of numbers
\end{Description}
%
\begin{Usage}
\begin{verbatim}
formatz_p(value)
\end{verbatim}
\end{Usage}
%
\begin{Arguments}
\begin{ldescription}
\item[\code{value}] a number or a vector of numbers to be formatted
\end{ldescription}
\end{Arguments}
%
\begin{Value}
a number or a vector of numbers formatted with 4 digits
\end{Value}
%
\begin{Examples}
\begin{ExampleCode}
formatz_p(c(1.000, 0.75643242, 0.000032431, 0.00214))

\end{ExampleCode}
\end{Examples}
\HeaderA{Kmax\_aim\_LB}{Function to print the histogram of the AIM::cv.cox.main output}{Kmax.Rul.aim.Rul.LB}
%
\begin{Description}
Function to print the histogram of the AIM::cv.cox.main output
\end{Description}
%
\begin{Usage}
\begin{verbatim}
Kmax_aim_LB(kmax.cycle = kmax.cycle)
\end{verbatim}
\end{Usage}
%
\begin{Arguments}
\begin{ldescription}
\item[\code{kmax.cycle}] The vector of values of the best biomarkers
\end{ldescription}
\end{Arguments}
%
\begin{Value}
an histogram
\end{Value}
\HeaderA{KM\_LB}{This function allows to create a KM survival curve overall or splitted by a categorical variable}{KM.Rul.LB}
%
\begin{Description}
This function allows to create a KM survival curve overall or splitted by a categorical variable
\end{Description}
%
\begin{Usage}
\begin{verbatim}
KM_LB(
  Event = "OS_EVENT",
  tEvent = "OS",
  strata = 1,
  data = data,
  title = "Prova",
  xlab = "Time in months",
  ylab = "Probaility of Surv",
  xlim = c(0, max(data[, tEvent], na.rm = T)),
  breaks_by = 3
)
\end{verbatim}
\end{Usage}
%
\begin{Arguments}
\begin{ldescription}
\item[\code{Event}] Event variable

\item[\code{tEvent}] Survival Time Variable

\item[\code{strata}] Variable to stratify (Default = 1)

\item[\code{data}] dataframe

\item[\code{title}] Graph title (Default = "Prova")

\item[\code{xlab}] x-axis title (Default = "Time in months")

\item[\code{ylab}] y-axis title (Default = "Probaility of Surv")

\item[\code{xlim}] limits of x-axis (Default is from 0 to maximum observed time)

\item[\code{breaks\_by}] breaks of risk table(Default = 3)
\end{ldescription}
\end{Arguments}
%
\begin{Value}
a KM graph
\end{Value}
\HeaderA{Lineplots\_LB}{Function to build the lineplots}{Lineplots.Rul.LB}
%
\begin{Description}
Function to build the lineplots
\end{Description}
%
\begin{Usage}
\begin{verbatim}
Lineplots_LB(
  data,
  variables,
  time,
  breaks = unique(data[, time]),
  label = unique(data[, time]),
  group = 1,
  col_lines = c("salmon", "royalblue"),
  stat_line = "median",
  smooth_line = FALSE,
  span_line = 0.3,
  lw_reg = 1,
  alpha_line = 1,
  ylim = c(0.2, 0.8),
  ribbon = TRUE,
  alpha_ribbon = 0.05,
  ID_lines = FALSE,
  ID = "ID",
  alpha_ID_line = 0.3,
  lw_ID_line = 0.2,
  Point = FALSE,
  alpha_point = 0.3,
  size_point = 0.3,
  col_title = FALSE,
  colour_title = NULL,
  size_title = 7,
  size_axis_x = 5,
  size_axis_y = 6,
  Overall = F,
  Test_results = Test_results,
  Posthoc = F,
  threshold_posthoc = 0.1,
  posthoc_test_size = 2,
  grid = T,
  ratio = 1,
  PPTX = F,
  pptx_width = 8.5,
  pptx_height = 5.5,
  target = paste0(path, "/file.pptx"),
  label_title = paste0("Lineplots by ", group, "\n", format(Sys.Date(), "%d/%m/%Y")),
  size_label_title = 2.5,
  extra = F,
  extra_text = NULL
)
\end{verbatim}
\end{Usage}
%
\begin{Arguments}
\begin{ldescription}
\item[\code{data}] A long-formatted dataframe

\item[\code{variables}] Vector of variables to plot

\item[\code{time}] Numeric variable to plot on the x-axis

\item[\code{breaks}] Numeric vector with x-axis breaks. Default: unique(data[, time])

\item[\code{label}] Vector with x-axis labels. Must be the same length of breaks. Default: unique(data[, time])

\item[\code{group}] Factor variable to group the lines

\item[\code{col\_lines}] Colours for the lines. Default: "salmon" \& "royalblue"

\item[\code{stat\_line}] Statistic method for the line. "median" or "mean".

\item[\code{smooth\_line}] Whether to show smooth or spline regression line. Defaul = FALSE

\item[\code{span\_line}] The span of smooth line. Defaul = 0.3

\item[\code{lw\_reg}] Linewidth for regression line

\item[\code{alpha\_line}] Alpha for regression line

\item[\code{ylim}] Limits for the y-axis to plot. Default: c(0.20, 0.80)

\item[\code{ribbon}] Whether to show or not ribbons

\item[\code{alpha\_ribbon}] Alpha for ribbons. Default: 0.05

\item[\code{ID\_lines}] Whether to show ID lines for every patient

\item[\code{ID}] ID variable. Default: "ID"

\item[\code{alpha\_ID\_line}] Alpha for ID lines. Default: 0.3

\item[\code{lw\_ID\_line}] Linewidth for ID lines. Default: 0.2

\item[\code{Point}] Whether to show points. Defaul = FALSE

\item[\code{alpha\_point}] Alpha points. Defaul = 0.3

\item[\code{size\_point}] Size points. Defaul = 0.3

\item[\code{col\_title}] Whether to personalize the colour of title. Default = FALSE

\item[\code{colour\_title}] A function to personalize the colour of title. See vignette for more.

\item[\code{size\_title}] Size of title. If grid recommended 7, if PPTX recommended 20

\item[\code{size\_axis\_x}] x-axis text size. If grid recommended 5, if PPTX recommended 14

\item[\code{size\_axis\_y}] y-axis text size. If grid recommended 6, if PPTX recommended 14

\item[\code{Overall}] Whether to add overall test. Default = FALSE. See vignette for more.

\item[\code{Test\_results}] Dataframe for overall and posthoc tests. See vignette for more.

\item[\code{Posthoc}] Whether to add posthoc tests with brackets.  Default = FALSE. See vignette for more.

\item[\code{threshold\_posthoc}] Threshold to display posthoc brackets

\item[\code{posthoc\_test\_size}] Size for annotations of posthoc p-values. Default = 2

\item[\code{grid}] Whether to build a grid pdf or a PPTX file. Default = TRUE

\item[\code{ratio}] Graph ratio when grid = TRUE

\item[\code{PPTX}] Whether to build PPTX or a grid pdf file. Default = FALSE. Must change grid = FALSE

\item[\code{pptx\_width}] Graph dimensions for PPTX in inches

\item[\code{pptx\_height}] Graph dimensions for PPTX in inches

\item[\code{target}] Path where to save the PPTX file

\item[\code{label\_title}] A title for your list. Default sets "Lineplots by grouping variable" and the current date

\item[\code{size\_label\_title}] Size for your list's title. Default: 2.5

\item[\code{extra}] Whether to add an extra text function. Default = FALSE

\item[\code{extra\_text}] A function to add extra functions to the graphs. See vignette for more.
\end{ldescription}
\end{Arguments}
%
\begin{Value}
When grid = TRUE returns a list of ggplots. When PPTX = TRUE and grid = FALSE returns a PPTX file in the target folder
\end{Value}
\HeaderA{LL\_Descrittive}{Function to build, starting from a dataset, the descriptive statistics of every variable}{LL.Rul.Descrittive}
%
\begin{Description}
Function to build, starting from a dataset, the descriptive statistics of every variable
\end{Description}
%
\begin{Usage}
\begin{verbatim}
LL_Descrittive(dataset, path = NULL)
\end{verbatim}
\end{Usage}
%
\begin{Arguments}
\begin{ldescription}
\item[\code{dataset}] dataframe

\item[\code{path}] where do you want it to be saved
\end{ldescription}
\end{Arguments}
\HeaderA{LL\_fisher\_gt\_flex}{Function to build coloumn marginal statistics and Fisher test}{LL.Rul.fisher.Rul.gt.Rul.flex}
%
\begin{Description}
Function to build coloumn marginal statistics and Fisher test
\end{Description}
%
\begin{Usage}
\begin{verbatim}
LL_fisher_gt_flex(data, row_var, col_var, label_row_var, label_col_var)
\end{verbatim}
\end{Usage}
%
\begin{Arguments}
\begin{ldescription}
\item[\code{data}] dataframe

\item[\code{row\_var}] row variable

\item[\code{col\_var}] column variable

\item[\code{label\_row\_var}] label for row

\item[\code{label\_col\_var}] label for column
\end{ldescription}
\end{Arguments}
\HeaderA{LL\_Npsurv\_format}{Function to get a cute format of npsurv output.}{LL.Rul.Npsurv.Rul.format}
%
\begin{Description}
Function to get a cute format of npsurv output.
\end{Description}
%
\begin{Usage}
\begin{verbatim}
LL_Npsurv_format(fit.npsurv)
\end{verbatim}
\end{Usage}
%
\begin{Arguments}
\begin{ldescription}
\item[\code{fit.npsurv}] A npsurv(Surv(time, event) \textasciitilde{} cov\_factor, data) object
\end{ldescription}
\end{Arguments}
%
\begin{Value}
A cute format of npsurv output
\end{Value}
\HeaderA{LL\_Tapply\_f}{Function for an easy application of the tapply}{LL.Rul.Tapply.Rul.f}
%
\begin{Description}
Function for an easy application of the tapply
\end{Description}
%
\begin{Usage}
\begin{verbatim}
LL_Tapply_f(data, var_quant, var_cat, digits = 2)
\end{verbatim}
\end{Usage}
%
\begin{Arguments}
\begin{ldescription}
\item[\code{data}] dataframe

\item[\code{var\_quant}] quantitative variable

\item[\code{var\_cat}] categorial variable

\item[\code{digits}] digits to display
\end{ldescription}
\end{Arguments}
\HeaderA{multivariate\_LL}{Function to create a multivariate cph model with a vector of variables}{multivariate.Rul.LL}
%
\begin{Description}
Function to create a multivariate cph model with a vector of variables
\end{Description}
%
\begin{Usage}
\begin{verbatim}
multivariate_LL(db, vars, ptime, pevent, dec_HR = 4)
\end{verbatim}
\end{Usage}
%
\begin{Arguments}
\begin{ldescription}
\item[\code{db}] A dataframe

\item[\code{vars}] Vector of variables to be included in the multivariate model

\item[\code{ptime}] Survival Time variable

\item[\code{pevent}] Event variable

\item[\code{dec\_HR}] digits of HR (Default = 4)
\end{ldescription}
\end{Arguments}
%
\begin{Value}
the multivariate model
\end{Value}
\HeaderA{New\_Project\_LB}{Function to create a new project in the default folder}{New.Rul.Project.Rul.LB}
%
\begin{Description}
Function to create a new project in the default folder
\end{Description}
%
\begin{Usage}
\begin{verbatim}
New_Project_LB(
  project_name,
  rstudio = rstudioapi::isAvailable(),
  open = rlang::is_interactive()
)
\end{verbatim}
\end{Usage}
%
\begin{Arguments}
\begin{ldescription}
\item[\code{project\_name}] The name of the Project

\item[\code{rstudio}] If `TRUE`, calls [use\_rstudio()] to make the new package or
project into an [RStudio
Project](https://r-pkgs.org/workflow101.html\#sec-workflow101-rstudio-projects).
If `FALSE` and a non-package project, a sentinel `.here` file is placed so
that the directory can be recognized as a project by the
[here](https://here.r-lib.org) or
[rprojroot](https://rprojroot.r-lib.org) packages.

\item[\code{open}] If `TRUE`, [activates][proj\_activate()] the new project:

* If using RStudio desktop, the package is opened in a new session.
* If on RStudio server, the current RStudio project is activated.
* Otherwise, the working directory and active project is changed.
\end{ldescription}
\end{Arguments}
%
\begin{Value}
Returns a folder in Projects with Analisi, Dati and Output subfolders
\end{Value}
\HeaderA{olink\_gsea\_map}{Function which prepares names for olink\_pathway\_enrichment}{olink.Rul.gsea.Rul.map}
%
\begin{Description}
This function changes the Assay column in order to map all the genes in the pathway enrichment analysis
\end{Description}
%
\begin{Usage}
\begin{verbatim}
olink_gsea_map(data, test_results)
\end{verbatim}
\end{Usage}
%
\begin{Arguments}
\begin{ldescription}
\item[\code{data}] NPX data frame in long format with at least protein name (Assay), OlinkID, UniProt,SampleID, QC\_Warning, NPX, and LOD

\item[\code{test\_results}] a dataframe of statistical test results including Adjusted\_pval and estimate columns.
\end{ldescription}
\end{Arguments}
%
\begin{Value}
the two data frames with Assay column adjusted for the pathway analysis
\end{Value}
\HeaderA{Outlier\_Report\_LB}{Function for outliers report in excel}{Outlier.Rul.Report.Rul.LB}
%
\begin{Description}
Function for outliers report in excel
\end{Description}
%
\begin{Usage}
\begin{verbatim}
Outlier_Report_LB(
  data,
  variables,
  ID = "ID",
  Group = "Time",
  k = 1.5,
  excel = FALSE,
  excel_path = "Output/Outlier Report.xlsx"
)
\end{verbatim}
\end{Usage}
%
\begin{Arguments}
\begin{ldescription}
\item[\code{data}] a dataframe

\item[\code{variables}] vector of variables to check

\item[\code{ID}] ID variable. Default = "ID"

\item[\code{Group}] Grouping variable. Default = "Time"

\item[\code{k}] Parameter between Q1-Q3 and IQR. Default = 1.5

\item[\code{excel}] Whether to create an excel file of the report

\item[\code{excel\_path}] Path for the excel file
\end{ldescription}
\end{Arguments}
%
\begin{Value}
a list of dataframe one of the global variable, the other for the variable stratified for the grouping variable
\end{Value}
\HeaderA{output.aim.f}{Function to print the output of the AIM function with Biomarker, Direction and Cutoff as a data frame model}{output.aim.f}
%
\begin{Description}
Function to print the output of the AIM function with Biomarker, Direction and Cutoff as a data frame model
\end{Description}
%
\begin{Usage}
\begin{verbatim}
output.aim.f(res.index, aim.data)
\end{verbatim}
\end{Usage}
%
\begin{Arguments}
\begin{ldescription}
\item[\code{res.index}] An output from the AIM package function

\item[\code{aim.data}] Data where the function was run on
\end{ldescription}
\end{Arguments}
%
\begin{Value}
A dataframe-like object
\end{Value}
\HeaderA{posthoc\_df\_LB}{Function for Boxplot\_LB}{posthoc.Rul.df.Rul.LB}
%
\begin{Description}
Function for Boxplot\_LB
\end{Description}
%
\begin{Usage}
\begin{verbatim}
posthoc_df_LB(Test_results, data, group, threshold_posthoc, i)
\end{verbatim}
\end{Usage}
%
\begin{Arguments}
\begin{ldescription}
\item[\code{Test\_results}] Test\_results

\item[\code{data}] data

\item[\code{group}] group

\item[\code{threshold\_posthoc}] threshold posthoc tests

\item[\code{i}] i
\end{ldescription}
\end{Arguments}
%
\begin{Value}
nothing
\end{Value}
\HeaderA{Posthoc\_lineplots\_LB}{Title}{Posthoc.Rul.lineplots.Rul.LB}
%
\begin{Description}
Title
\end{Description}
%
\begin{Usage}
\begin{verbatim}
Posthoc_lineplots_LB(Test_results, data, time, threshold_posthoc, i)
\end{verbatim}
\end{Usage}
%
\begin{Arguments}
\begin{ldescription}
\item[\code{Test\_results}] Test\_results

\item[\code{data}] data

\item[\code{threshold\_posthoc}] Threshold ph tests

\item[\code{i}] i

\item[\code{Time}] Time
\end{ldescription}
\end{Arguments}
\HeaderA{Print\_LB}{Function to print the PDF with the grid.arrange function}{Print.Rul.LB}
%
\begin{Description}
Function to print the PDF with the grid.arrange function
\end{Description}
%
\begin{Usage}
\begin{verbatim}
Print_LB(
  plot_list,
  path_print = path_print,
  nrow = 8,
  ncol = 6,
  ext = c("pdf", "svg", "png", "emf", "tiff", "jpeg"),
  width_pg = 21,
  height_pg = 29.7,
  return = FALSE
)
\end{verbatim}
\end{Usage}
%
\begin{Arguments}
\begin{ldescription}
\item[\code{plot\_list}] The list you want to be plotted

\item[\code{path\_print}] The path where you want your PDF to be printed

\item[\code{nrow}] Rows of your grid

\item[\code{ncol}] Columns of your grid

\item[\code{ext}] File extention

\item[\code{width\_pg}] page width in cm

\item[\code{height\_pg}] page height in cm

\item[\code{return}] if you want to assign your grid
\end{ldescription}
\end{Arguments}
%
\begin{Value}
A pdf in the path\_output
\end{Value}
\HeaderA{Pushover\_LB}{Sending a Pushover notification on your device}{Pushover.Rul.LB}
%
\begin{Description}
Sending a Pushover notification on your device
\end{Description}
%
\begin{Usage}
\begin{verbatim}
Pushover_LB(
  dest = "both",
  script = 0,
  timestamp = TRUE,
  priority = 0,
  app = "ayje1n4x8fi64bupdn5shjnwd8ut95",
  title = "RStudio",
  attachment = NULL,
  start_time = NULL
)
\end{verbatim}
\end{Usage}
%
\begin{Arguments}
\begin{ldescription}
\item[\code{dest}] Who is going to receive the notification (Default = both)

\item[\code{script}] Text of your message

\item[\code{timestamp}] If you want the time printed in your notification, if TRUE it requires a start\_time in the .GlobalEnv

\item[\code{priority}] Priority of your notification (-2; 2)

\item[\code{app}] Your app API key

\item[\code{title}] Your title notification

\item[\code{attachment}] Path to an image, if you want to attach it

\item[\code{start\_time}] Start time (Default = NULL)
\end{ldescription}
\end{Arguments}
\HeaderA{Stringa\_LL}{Funzione che riceve in input le posizioni dei nomi di un dataframe e crea una stringa di tali nomi separati da virgola o da altro segno/simbolo}{Stringa.Rul.LL}
%
\begin{Description}
Funzione che riceve in input le posizioni dei nomi di un dataframe e crea una stringa di tali nomi separati da virgola o da altro segno/simbolo
\end{Description}
%
\begin{Usage}
\begin{verbatim}
Stringa_LL(data, vet, sep = ",")
\end{verbatim}
\end{Usage}
%
\begin{Arguments}
\begin{ldescription}
\item[\code{data}] datafrane

\item[\code{vet}] vector of positions for names in the dataset

\item[\code{sep}] symbol to separate names from each other
\end{ldescription}
\end{Arguments}
\HeaderA{Sys\_Time\_LB}{Function to get the Sys.time() in a cute and nice format}{Sys.Rul.Time.Rul.LB}
%
\begin{Description}
Function to get the Sys.time() in a cute and nice format
\end{Description}
%
\begin{Usage}
\begin{verbatim}
Sys_Time_LB()
\end{verbatim}
\end{Usage}
%
\begin{Value}
The Sys.time() in a cute format
\end{Value}
\HeaderA{telegram\_mess\_LB}{Function to send a Telegram message with BiostatUO9 bot. NB: must create a start\_time before running it}{telegram.Rul.mess.Rul.LB}
%
\begin{Description}
Function to send a Telegram message with BiostatUO9 bot. NB: must create a start\_time before running it
\end{Description}
%
\begin{Usage}
\begin{verbatim}
telegram_mess_LB(
  dest = "both",
  script = 0,
  rm_start_time = TRUE,
  timestamp = TRUE
)
\end{verbatim}
\end{Usage}
%
\begin{Arguments}
\begin{ldescription}
\item[\code{dest}] Who is going to receive the message

\item[\code{script}] The title of the message

\item[\code{rm\_start\_time}] If you want the start\_time item to be removed after the message is sent

\item[\code{timestamp}] Do you want the time in your message
\end{ldescription}
\end{Arguments}
%
\begin{Value}
Nothing
\end{Value}
\HeaderA{univariate\_LL}{This function allows you to create the univariate regression model for a vector of variables}{univariate.Rul.LL}
%
\begin{Description}
This function allows you to create the univariate regression model for a vector of variables
\end{Description}
%
\begin{Usage}
\begin{verbatim}
univariate_LL(db, vars, ptime, pevent, dec_HR = 4)
\end{verbatim}
\end{Usage}
%
\begin{Arguments}
\begin{ldescription}
\item[\code{db}] dataframe

\item[\code{vars}] vector with variables name

\item[\code{ptime}] Survival Time variable

\item[\code{pevent}] Event variable

\item[\code{dec\_HR}] digits of HR (Default = 4)
\end{ldescription}
\end{Arguments}
%
\begin{Value}
a dataframe with all univariate models
\end{Value}
\HeaderA{vett.quoted}{Funzione che permette partendo da un vettore, di riscrivere quel vettore in varie forme}{vett.quoted}
%
\begin{Description}
Funzione che permette partendo da un vettore, di riscrivere quel vettore in varie forme
\end{Description}
%
\begin{Usage}
\begin{verbatim}
vett.quoted(vettore, sym = ", ", quote = T)
\end{verbatim}
\end{Usage}
%
\begin{Arguments}
\begin{ldescription}
\item[\code{vettore}] Starting vector

\item[\code{sym}] Symbol of separation (Default ", ")

\item[\code{quote}] Vector elements to be quoted or not (Default = T)
\end{ldescription}
\end{Arguments}
%
\begin{Value}
Una stringa di elementi formattati al meglio
\end{Value}
\printindex{}
\end{document}
