\nonstopmode{}
\documentclass[a4paper]{book}
\usepackage[times,inconsolata,hyper]{Rd}
\usepackage{makeidx}
\makeatletter\@ifl@t@r\fmtversion{2018/04/01}{}{\usepackage[utf8]{inputenc}}\makeatother
% \usepackage{graphicx} % @USE GRAPHICX@
\makeindex{}
\begin{document}
\chapter*{}
\begin{center}
{\textbf{\huge Package `LandS'}}
\par\bigskip{\large \today}
\end{center}
\ifthenelse{\boolean{Rd@use@hyper}}{\hypersetup{pdftitle = {LandS: Biostatistic Tools for DCP}}}{}
\begin{description}
\raggedright{}
\item[Type]\AsIs{Package}
\item[Title]\AsIs{Biostatistic Tools for DCP}
\item[Version]\AsIs{1.0.1}
\item[Author]\AsIs{Stefano Bergamini and Luca Lalli }
\item[Maintainer]\AsIs{Nobody OfCourse }\email{stefano.bergamini@istitutotumori.mi.it}\AsIs{}
\item[Description]\AsIs{This package provides useful functions in daily clinical pratice for biostatistics}
\item[License]\AsIs{No you can steal it when you want}
\item[Encoding]\AsIs{UTF-8}
\item[LazyData]\AsIs{true}
\item[RoxygenNote]\AsIs{7.3.2}
\end{description}
\Rdcontents{Contents}
\HeaderA{Boxplot\_LB}{This function creates a list of boxplot}{Boxplot.Rul.LB}
%
\begin{Description}
This function creates a list of boxplot
\end{Description}
%
\begin{Usage}
\begin{verbatim}
Boxplot_LB(
  data,
  variables,
  group,
  ID_lines = FALSE,
  Posthoc = FALSE,
  Point = F,
  Median_line = F,
  rm.outliers = F,
  alpha_box = 0.1,
  width_box = 0.2,
  size_median_line = 0.8,
  lwd_box = 0.1,
  lwd_ID_line = 0.2,
  alpha_ID_line = 0.3,
  alpha_point = 0.3,
  size_point = 0.3,
  Test_results = NULL,
  threshold_posthoc = 0.1,
  axis_y_title = NULL,
  axis_x_title = NULL,
  size_axis_x = 6,
  size_axis_y = 6,
  ID = "ID",
  legend_cod = NULL,
  breaks_axis_x = levels(data[, group]),
  labels_axis_x = levels(data[, group]),
  grid = TRUE,
  PPTX = FALSE,
  pptx_width = 7.5,
  pptx_height = 5.5,
  extra = F,
  extra_text = NULL,
  palette_boxplot = c("salmon", "royalblue", "forestgreen", "gold"),
  palette_title = "black",
  size_title = 12,
  target = paste0(path_out, "/Boxplot.pptx"),
  ratio = 1,
  telegram = "none"
)
\end{verbatim}
\end{Usage}
%
\begin{Arguments}
\begin{ldescription}
\item[\code{data}] dataframe

\item[\code{variables}] vector containing all variables of interest

\item[\code{group}] factor variable splitting the data

\item[\code{ID\_lines}] whether to print the lines for paired observations

\item[\code{Posthoc}] whether to display Posthoc tests

\item[\code{Point}] whether to display observation points

\item[\code{Median\_line}] whether to display the line connecting medians

\item[\code{rm.outliers}] whether to remove outliers from display

\item[\code{alpha\_box}] alpha parameter for the boxes

\item[\code{width\_box}] box's width

\item[\code{size\_median\_line}] median linewidth

\item[\code{lwd\_box}] box linewidth

\item[\code{lwd\_ID\_line}] linewidth for paired observations

\item[\code{alpha\_ID\_line}] alpha for paired observations

\item[\code{alpha\_point}] alpha for points

\item[\code{size\_point}] size for points

\item[\code{Test\_results}] dataframe for global and posthoc tests, see cont\_var\_test\_LB

\item[\code{threshold\_posthoc}] threshold for displaying posthoc tests

\item[\code{size\_axis\_x}] axis y title dimension

\item[\code{size\_axis\_y}] axis x title dimension

\item[\code{ID}] ID variable

\item[\code{telegram}] 
\end{ldescription}
\end{Arguments}
%
\begin{Value}
Una lista di boxplot
\end{Value}
\HeaderA{cont\_var\_test\_LB}{Test for continuous variables splitted by categories}{cont.Rul.var.Rul.test.Rul.LB}
%
\begin{Description}
The most powerful function ever created. You can perform the 4 major tests and the posthoc tests for Friedman and Kruskal-Wallis.
If you are dumb (option dumb = T) you can also perform posthoc tests without correcting for test multiplicity.
Please do not try this at home/work and consider asking a statistician before performing any test.
\end{Description}
%
\begin{Usage}
\begin{verbatim}
cont_var_test_LB(
  data,
  variables,
  paired = FALSE,
  group,
  dumb = FALSE,
  ID = "ID",
  num_dec = 2,
  excel = F,
  excel_path = paste0(path_out, "/Results.xlsx"),
  telegram = "none"
)
\end{verbatim}
\end{Usage}
%
\begin{Arguments}
\begin{ldescription}
\item[\code{data}] dataframe

\item[\code{variables}] vector containing all variables of interest

\item[\code{paired}] FALSE/TRUE

\item[\code{group}] factor variable splitting the data

\item[\code{dumb}] FALSE are you dumb? Hope not

\item[\code{ID}] ID variabl (Default = "ID")

\item[\code{num\_dec}] Decimal number for mean and SD (Default = 2)

\item[\code{excel}] export fuction results as multiple Excel sheets

\item[\code{excel\_path}] path where you want your Excel

\item[\code{telegram}] send a telegram message
\end{ldescription}
\end{Arguments}
%
\begin{Value}
Una lista con dataset
\end{Value}
%
\begin{Examples}
\begin{ExampleCode}
cont_var_test_LB(data = iris, variables = c("Sepal.Length", "Sepal.Width"), group = "Species", paired = F)
\end{ExampleCode}
\end{Examples}
\HeaderA{correlazioni\_LB}{This function computes the correlation coefficients and prints the pairs from the heightest coefficient}{correlazioni.Rul.LB}
%
\begin{Description}
This function computes the correlation coefficients and prints the pairs from the heightest coefficient
\end{Description}
%
\begin{Usage}
\begin{verbatim}
correlazioni_LB(
  dataset,
  lista_vars,
  method = "spearman",
  rho_dec = 3,
  pval_dec = 4
)
\end{verbatim}
\end{Usage}
%
\begin{Arguments}
\begin{ldescription}
\item[\code{dataset}] dataframe

\item[\code{lista\_vars}] vector of numeric variables to be computed the correlation

\item[\code{method}] method to compute the correlation coefficient (Default = "spearman")

\item[\code{rho\_dec}] number of decimal for rho (Default = 3)

\item[\code{pval\_dec}] number of decimal for the pvalue (Default = 4)
\end{ldescription}
\end{Arguments}
%
\begin{Value}
Una lista con dataset
\end{Value}
\HeaderA{filename\_LB}{This function returns the filename to be outputted}{filename.Rul.LB}
%
\begin{Description}
This function returns the filename to be outputted
\end{Description}
%
\begin{Usage}
\begin{verbatim}
filename_LB(
  filename = "Prova",
  extention = ".png",
  output = path_output,
  datetime = F
)
\end{verbatim}
\end{Usage}
%
\begin{Arguments}
\begin{ldescription}
\item[\code{filename}] name of the file

\item[\code{extention}] file extention

\item[\code{output}] the main output path

\item[\code{datetime}] whether to print the datetime in a cute format
\end{ldescription}
\end{Arguments}
\HeaderA{formatz\_p}{Function to get a formatted p-value for a number o a vector of numbers}{formatz.Rul.p}
%
\begin{Description}
Function to get a formatted p-value for a number o a vector of numbers
\end{Description}
%
\begin{Usage}
\begin{verbatim}
formatz_p(value)
\end{verbatim}
\end{Usage}
%
\begin{Arguments}
\begin{ldescription}
\item[\code{value}] a number or a vector of numbers to be formatted
\end{ldescription}
\end{Arguments}
%
\begin{Value}
a number or a vector of numbers formatted with 4 digits
\end{Value}
%
\begin{Examples}
\begin{ExampleCode}
formatz_p(c(1.000, 0.75643242, 0.000032431, 0.00214))

\end{ExampleCode}
\end{Examples}
\HeaderA{Kmax\_aim\_LB}{Function to print the histogram of the AIM::cv.cox.main output}{Kmax.Rul.aim.Rul.LB}
%
\begin{Description}
Function to print the histogram of the AIM::cv.cox.main output
\end{Description}
%
\begin{Usage}
\begin{verbatim}
Kmax_aim_LB(kmax.cycle = kmax.cycle)
\end{verbatim}
\end{Usage}
%
\begin{Arguments}
\begin{ldescription}
\item[\code{kmax.cycle}] The vector of values of the best biomarkers
\end{ldescription}
\end{Arguments}
%
\begin{Value}
an histogram
\end{Value}
\HeaderA{KM\_LB}{This function allows to create a KM survival curve overall or splitted by a categorical variable}{KM.Rul.LB}
%
\begin{Description}
This function allows to create a KM survival curve overall or splitted by a categorical variable
\end{Description}
%
\begin{Usage}
\begin{verbatim}
KM_LB(
  Event = "OS_EVENT",
  tEvent = "OS",
  strata = 1,
  data = data,
  title = "Prova",
  xlab = "Time in months",
  ylab = "Probaility of Surv",
  xlim = c(0, max(data[, tEvent], na.rm = T)),
  breaks_by = 3
)
\end{verbatim}
\end{Usage}
%
\begin{Arguments}
\begin{ldescription}
\item[\code{Event}] Event variable

\item[\code{tEvent}] Survival Time Variable

\item[\code{strata}] Variable to stratify (Default = 1)

\item[\code{data}] dataframe

\item[\code{title}] Graph title (Default = "Prova")

\item[\code{xlab}] x-axis title (Default = "Time in months")

\item[\code{ylab}] y-axis title (Default = "Probaility of Surv")

\item[\code{xlim}] limits of x-axis (Default is from 0 to maximum observed time)

\item[\code{breaks\_by}] breaks of risk table(Default = 3)
\end{ldescription}
\end{Arguments}
%
\begin{Value}
a KM graph
\end{Value}
\HeaderA{Lineplots\_LB}{Function to build the lineplots}{Lineplots.Rul.LB}
%
\begin{Description}
Function to build the lineplots
\end{Description}
%
\begin{Usage}
\begin{verbatim}
Lineplots_LB(
  data,
  variables,
  time,
  group = 1,
  split = F,
  lw_reg = 1,
  size_point = 0.6,
  size_title = 12,
  col_title = 1,
  size_axis_x = 5,
  size_axis_y = 6,
  size_title_grid = 7,
  breaks = unique(data[, time]),
  label = unique(data[, time]),
  ylim = c(0.2, 0.8),
  Posthoc = F,
  Friedman = F,
  Test_results = Test_results,
  grid = T,
  ratio = 1,
  extra = F,
  extra_text = NULL,
  PPTX = F,
  pptx_width = 8.5,
  pptx_height = 5.5,
  threshold_posthoc = 0.1,
  check = F,
  target = paste0(path, "/file.pptx"),
  col_lines = c("salmon", "royalblue")
)
\end{verbatim}
\end{Usage}
%
\begin{Arguments}
\begin{ldescription}
\item[\code{data}] dataset

\item[\code{variables}] vector of variables

\item[\code{time}] x-axis variable

\item[\code{group}] factor variable to group

\item[\code{split}] whether to split in two windows the lines

\item[\code{lw\_reg}] lwd of regression line

\item[\code{size\_point}] size of points

\item[\code{size\_title}] size of title

\item[\code{col\_title}] colour of title

\item[\code{size\_axis\_x}] x-axis text size

\item[\code{size\_axis\_y}] y-axis text size

\item[\code{size\_title\_grid}] size of title in the grid

\item[\code{breaks}] breaks of x-axis

\item[\code{label}] labels of x-axis

\item[\code{ylim}] ylim to display in the graph

\item[\code{Posthoc}] whether to display posthoc tests

\item[\code{Friedman}] friedman overall test dataset

\item[\code{Test\_results}] posthoc test dataset

\item[\code{grid}] whether to build a grid or a pptx

\item[\code{ratio}] graph ratio

\item[\code{extra}] do you want to add extra option?

\item[\code{extra\_text}] write your additional options

\item[\code{PPTX}] whether to build a pptx or a grid

\item[\code{pptx\_width}] inch

\item[\code{pptx\_height}] inch

\item[\code{threshold\_posthoc}] threshold to display posthoc brackets

\item[\code{check}] check the correctness of your graph

\item[\code{target}] where do you want your pptx to be saved

\item[\code{col\_lines}] splitted lines colour
\end{ldescription}
\end{Arguments}
\HeaderA{LL\_Descrittive}{Function to build, starting from a dataset, the descriptive statistics of every variable}{LL.Rul.Descrittive}
%
\begin{Description}
Function to build, starting from a dataset, the descriptive statistics of every variable
\end{Description}
%
\begin{Usage}
\begin{verbatim}
LL_Descrittive(dataset, path = NULL)
\end{verbatim}
\end{Usage}
%
\begin{Arguments}
\begin{ldescription}
\item[\code{dataset}] dataframe

\item[\code{path}] where do you want it to be saved
\end{ldescription}
\end{Arguments}
\HeaderA{LL\_fisher\_gt\_flex}{Function to build coloumn marginal statistics and Fisher test}{LL.Rul.fisher.Rul.gt.Rul.flex}
%
\begin{Description}
Function to build coloumn marginal statistics and Fisher test
\end{Description}
%
\begin{Usage}
\begin{verbatim}
LL_fisher_gt_flex(data, row_var, col_var, label_row_var, label_col_var)
\end{verbatim}
\end{Usage}
%
\begin{Arguments}
\begin{ldescription}
\item[\code{data}] dataframe

\item[\code{row\_var}] row variable

\item[\code{col\_var}] column variable

\item[\code{label\_row\_var}] label for row

\item[\code{label\_col\_var}] label for column
\end{ldescription}
\end{Arguments}
\HeaderA{LL\_Npsurv\_format}{Function to get a cute format of npsurv output.}{LL.Rul.Npsurv.Rul.format}
%
\begin{Description}
Function to get a cute format of npsurv output.
\end{Description}
%
\begin{Usage}
\begin{verbatim}
LL_Npsurv_format(fit.npsurv)
\end{verbatim}
\end{Usage}
%
\begin{Arguments}
\begin{ldescription}
\item[\code{fit.npsurv}] A npsurv(Surv(time, event) \textasciitilde{} cov\_factor, data) object
\end{ldescription}
\end{Arguments}
%
\begin{Value}
A cute format of npsurv output
\end{Value}
\HeaderA{LL\_Tapply\_f}{Function for an easy application of the tapply}{LL.Rul.Tapply.Rul.f}
%
\begin{Description}
Function for an easy application of the tapply
\end{Description}
%
\begin{Usage}
\begin{verbatim}
LL_Tapply_f(data, var_quant, var_cat, digits = 2)
\end{verbatim}
\end{Usage}
%
\begin{Arguments}
\begin{ldescription}
\item[\code{data}] dataframe

\item[\code{var\_quant}] quantitative variable

\item[\code{var\_cat}] categorial variable

\item[\code{digits}] digits to display
\end{ldescription}
\end{Arguments}
\HeaderA{multivariate\_LL}{Function to create a multivariate cph model with a vector of variables}{multivariate.Rul.LL}
%
\begin{Description}
Function to create a multivariate cph model with a vector of variables
\end{Description}
%
\begin{Usage}
\begin{verbatim}
multivariate_LL(db, vars, ptime, pevent, dec_HR = 4)
\end{verbatim}
\end{Usage}
%
\begin{Arguments}
\begin{ldescription}
\item[\code{db}] A dataframe

\item[\code{vars}] Vector of variables to be included in the multivariate model

\item[\code{ptime}] Survival Time variable

\item[\code{pevent}] Event variable

\item[\code{dec\_HR}] digits of HR (Default = 4)
\end{ldescription}
\end{Arguments}
%
\begin{Value}
the multivariate model
\end{Value}
\HeaderA{New\_Project\_LB}{Function to create a new project in the default folder}{New.Rul.Project.Rul.LB}
%
\begin{Description}
Function to create a new project in the default folder
\end{Description}
%
\begin{Usage}
\begin{verbatim}
New_Project_LB(name_project, pc = c("Luca", "Stefano"))
\end{verbatim}
\end{Usage}
%
\begin{Arguments}
\begin{ldescription}
\item[\code{name\_project}] The name of the Project

\item[\code{pc}] Which pc are we operating
\end{ldescription}
\end{Arguments}
%
\begin{Value}
Returns a folder in Projects with Analisi, Dati and Output subfolders
\end{Value}
\HeaderA{output.aim.f}{Function to print the output of the AIM function with Biomarker, Direction and Cutoff as a data frame model}{output.aim.f}
%
\begin{Description}
Function to print the output of the AIM function with Biomarker, Direction and Cutoff as a data frame model
\end{Description}
%
\begin{Usage}
\begin{verbatim}
output.aim.f(res.index, aim.data)
\end{verbatim}
\end{Usage}
%
\begin{Arguments}
\begin{ldescription}
\item[\code{res.index}] An output from the AIM package function

\item[\code{aim.data}] Data where the function was run on
\end{ldescription}
\end{Arguments}
%
\begin{Value}
A dataframe-like object
\end{Value}
\HeaderA{PDF\_print\_LB}{Function to print the PDF with the grid.arrange function}{PDF.Rul.print.Rul.LB}
%
\begin{Description}
Function to print the PDF with the grid.arrange function
\end{Description}
%
\begin{Usage}
\begin{verbatim}
PDF_print_LB(
  plot_list,
  path_print = path_print,
  nrow = 8,
  ncol = 6,
  variables = vett_all_markers
)
\end{verbatim}
\end{Usage}
%
\begin{Arguments}
\begin{ldescription}
\item[\code{plot\_list}] The list you want to be plotted

\item[\code{path\_print}] The path where you want your PDF to be printed

\item[\code{nrow}] Rows of your grid

\item[\code{ncol}] Columns of your grid

\item[\code{variables}] Number of total graphs to be printed
\end{ldescription}
\end{Arguments}
%
\begin{Value}
A pdf in the path\_output
\end{Value}
\HeaderA{Stringa\_LL}{Funzione che riceve in input le posizioni dei nomi di un dataframe e crea una stringa di tali nomi separati da virgola o da altro segno/simbolo}{Stringa.Rul.LL}
%
\begin{Description}
Funzione che riceve in input le posizioni dei nomi di un dataframe e crea una stringa di tali nomi separati da virgola o da altro segno/simbolo
\end{Description}
%
\begin{Usage}
\begin{verbatim}
Stringa_LL(data, vet, sep = ",")
\end{verbatim}
\end{Usage}
%
\begin{Arguments}
\begin{ldescription}
\item[\code{data}] datafrane

\item[\code{vet}] vector of positions for names in the dataset

\item[\code{sep}] symbol to separate names from each other
\end{ldescription}
\end{Arguments}
\HeaderA{Sys\_Time\_LB}{Function to get the Sys.time() in a cute and nice format}{Sys.Rul.Time.Rul.LB}
%
\begin{Description}
Function to get the Sys.time() in a cute and nice format
\end{Description}
%
\begin{Usage}
\begin{verbatim}
Sys_Time_LB()
\end{verbatim}
\end{Usage}
%
\begin{Value}
The Sys.time() in a cute format
\end{Value}
\HeaderA{telegram\_mess\_LB}{Function to send a Telegram message with BiostatUO9 bot. NB: must create a start\_time before running it}{telegram.Rul.mess.Rul.LB}
%
\begin{Description}
Function to send a Telegram message with BiostatUO9 bot. NB: must create a start\_time before running it
\end{Description}
%
\begin{Usage}
\begin{verbatim}
telegram_mess_LB(
  process_time = {
    
    format(lubridate::seconds_to_period(round(as.numeric(difftime(Sys.time(), start_time,
    units = "secs")))), "%H:%M:%S")
 },
  dest = "both",
  script = 0,
  rm_start_time = T
)
\end{verbatim}
\end{Usage}
%
\begin{Arguments}
\begin{ldescription}
\item[\code{process\_time}] Just don't modify it

\item[\code{dest}] Who is going to receive the message

\item[\code{script}] The title of the message

\item[\code{rm\_start\_time}] If you want the start\_time item to be removed after the message is sent
\end{ldescription}
\end{Arguments}
%
\begin{Value}
Nothing
\end{Value}
\HeaderA{univariate\_LL}{This function allows you to create the univariate regression model for a vector of variables}{univariate.Rul.LL}
%
\begin{Description}
This function allows you to create the univariate regression model for a vector of variables
\end{Description}
%
\begin{Usage}
\begin{verbatim}
univariate_LL(db, vars, ptime, pevent, dec_HR = 4)
\end{verbatim}
\end{Usage}
%
\begin{Arguments}
\begin{ldescription}
\item[\code{db}] dataframe

\item[\code{vars}] vector with variables name

\item[\code{ptime}] Survival Time variable

\item[\code{pevent}] Event variable

\item[\code{dec\_HR}] digits of HR (Default = 4)
\end{ldescription}
\end{Arguments}
%
\begin{Value}
a dataframe with all univariate models
\end{Value}
\HeaderA{vett.quoted}{Funzione che permette partendo da un vettore, di riscrivere quel vettore in varie forme}{vett.quoted}
%
\begin{Description}
Funzione che permette partendo da un vettore, di riscrivere quel vettore in varie forme
\end{Description}
%
\begin{Usage}
\begin{verbatim}
vett.quoted(vettore, sym = ", ", quote = T)
\end{verbatim}
\end{Usage}
%
\begin{Arguments}
\begin{ldescription}
\item[\code{vettore}] Starting vector

\item[\code{sym}] Symbol of separation (Default ", ")

\item[\code{quote}] Vector elements to be quoted or not (Default = T)
\end{ldescription}
\end{Arguments}
%
\begin{Value}
Una stringa di elementi formattati al meglio
\end{Value}
\printindex{}
\end{document}
